\documentclass[12pt]{article}
\usepackage{amsmath}

\title {Jacobian Free Newton's Method for Non-Linear Systems with Composit Residual Functions via MAUD}

\begin{document}

\maketitle

\section{Problem Definition}

Composite resids arise when solving a non-linear system whos residual evaluation is a function of two
or more different components. For example, consider a non linear system with state variable vector, $u$, and
residual vector, $r$, computed via two components, $Y$ and $R$. Assume $u$ and $r$ are of size $n$.

\begin{equation}
  u \longrightarrow Y \longrightarrow y \longrightarrow R \longrightarrow r
\end{equation}

$y$ and $r$ are explicit functions of $u$ and $y$ respectively

\begin{align}
  y = & Y(u) \notag \\
  r = & R(y)
\end{align}

A composite residual function, $G$, is then defined

\begin{equation}
  G(u) = R(Y(u))
\end{equation}

The problem definition is then given as follows

\begin{align}
  Find &\  u \notag \\
  s.t. &\  G(u) = 0
\end{align}

\section{Traditional Newtons Method}

The Newton update, $\Delta u$, is computed via
\begin{equation}
  \frac{\partial G}{\partial u} \Delta u = -G(u)
\end{equation}

Expanding the partial derivative of $G$ out to account for the composite residual function
gives

\begin{equation}
  \frac{\partial R}{\partial y}  \frac{\partial Y}{\partial u} \Delta u = -R\left(Y(u)\right)
  \label{expanded_newton_iter}
\end{equation}

In order to compute $\Delta u$, the partial derivatives
$\frac{\partial R}{\partial y}$ and  $\frac{\partial Y}{\partial u}$ must first be computed.
Then a matrix multiplication is performed ($\mathcal{O}(n^3)$) to get $\frac{\partial G}{\partial u}$.
Lastly the solution to a size $n$ linear system is found.

\section{MAUD Formulation}

Representing the composite residual as a system of non linear equations using the MAUD yeilds
two separate, but coupled, residual functions with $y$ and $r$ as state variables
and $u$ as an input.

\begin{align}
  A(u,y,r) & = y - Y(u) \label{MAUD_y} \\
  B(u,y,r) & = r - R(y) \label{MAUD_r}
  \label{maud1}
\end{align}

The original problem formulation seeks to drive $r$ to 0. So we'll just assume
that $r$ is 0, then find values of $u$ and $y$ that drive
the remaining resids to 0. This gives a new system of equations:

\begin{align}
  C(u,y) & = y - Y(u) = 0 \label{y_resid} \\
  D(r,y) & = R(y) = 0 \label{r_resid}
  \label{maud2}
\end{align}

You can compute a Newton update for the non-linear system via

\begin{equation}
  \begin{bmatrix}
    \frac{\partial C}{\partial u} & \frac{\partial C}{\partial y} \\
    \frac{\partial D}{\partial u} & \frac{\partial D}{\partial y}
  \end{bmatrix}
  \begin{bmatrix}
    \Delta u \\
    \Delta y
  \end{bmatrix}
  =
  \begin{bmatrix}
    -C(u,y) \\
    -D(u,y)
  \end{bmatrix}
\end{equation}

\begin{equation}
  \begin{bmatrix}
    -\frac{\partial Y}{\partial u} & 1 \\
    0 & \frac{\partial R}{\partial y}
  \end{bmatrix}
  \begin{bmatrix}
    \Delta u \\
    \Delta y \\
  \end{bmatrix}
  =
  \begin{bmatrix}
    Y(u) - y \\
    - R(y)
  \end{bmatrix}
\end{equation}

Expanding out the linear system yields

\begin{align}
  -\frac{\partial Y}{\partial u} \Delta u + \Delta y & = Y(u) - y   \label{Delta_u} \\
  \frac{\partial R}{\partial y} \Delta y & = -R(y)   \label{Delta_y}
\end{align}


Multiplying both sides by of Eqn.~\ref{Delta_u} by $\frac{\partial R}{\partial y}$ yields

\begin{equation}
  -\frac{\partial R}{\partial y} \frac{\partial Y}{\partial u} \Delta u + \frac{\partial R}{\partial y} \Delta y = Y(u) - y
\end{equation}

Now, assuming that you always executed $Y$ first and then $R$, you know that $Y(u) - y$ will always be identically 0.

\begin{equation}
  -\frac{\partial R}{\partial y} \frac{\partial Y}{\partial u} \Delta u + \frac{\partial R}{\partial y} \Delta y = 0
  \label{newton_update_1}
\end{equation}

Finally, substituting Eqn.~\ref{Delta_y} into Eqn.~\ref{newton_update_1} yeilds

\begin{equation}
  \frac{\partial R}{\partial y} \frac{\partial Y}{\partial u} \Delta u = -R(y)
  \label{maud_Delta_u_update}
\end{equation}

We now note that Eqn.~\ref{maud_Delta_u_update} and Eqn.~\ref{expanded_newton_iter} are equivalent, hence
the resulting $\Delta u$ value computed via the traditional method and the MAUD method are also
equivalent.

However, the MAUD provides a computational savings. The partial derivatives $\frac{\partial R}{\partial y}$
and $\frac{\partial Y}{\partial u}$ must still be computed. But the newton update, $\Delta u$ can now be found
directly via a single solution to a linear system of size $n$ without the expensive matrix
multiplication step needed with the traditional method.

\end{document}
