\documentclass[12pt]{article}
\usepackage{amsmath}
\usepackage{tikz}
\usepackage[simplified]{pgf-umlcd}
\usepackage{parskip}
\usepackage{verbments}


\usepackage[letterpaper]{geometry}
\geometry{left=.5in, right=.5in}



\newcommand{\classname}[1]{\texttt{#1}}
\newcommand{\method}[1]{\texttt{#1}}
\newcommand{\var}[1]{\texttt{#1}}


\title {OpenMDAO 1.0 Design Doc}

\begin{document}
\maketitle

\section{Analysis Composition Objects}

OpenMDAO is designed around the Modular Analysis and Unfied Derivatives (MAUD) mathematical
architecture. In MAUD, all analyses are represented as systems of coupled non-linear implicit equations.
\classname{System} is the baseclass for all objects that are responsible for mathematical computations.
Any instance of \classname{System} must implement 5 methods in order to be able to perform
analysis:

\begin{itemize}
    \item apply\_nonlinear: solve for the residual values given the values for
    the inputs and states
    \item solve\_nonlienar: given values for the inputs, find the values of the
    state variables that drive the non linear residual equations to zero
    \item linearize: compute partial derivatives around current location for all
    residuals as a function of current input and state variables
    \item multi\_apply\_linear: linear operator which provides the product of the systems
    Jacobian with a given set of vectors
    \item multi\_solve\_linear: solve the linear system defined by the jacobian on the
    lhs and a set of vectors on the right hand side
\end{itemize}

There are three different types of \classname{System} that the user can user or extend:
\begin{enumerate}
    \item Component: responsible creating variables, performing computations with them, and providing derivatives of those computations
    \item Group: Defined collection of systems, executed in a given order, that are solved (linear and non-linear) as a block
    \item Namespace: A group that also defines a unique name-space for the systems it contains and tracks data connections between its children
\end{enumerate}


\begin{tikzpicture}[]% [ show background grid ]
    \begin{interface}[text width =10cm ]{System}{0,0}
        \operation{apply\_nonlinear(pvec, uvec, fvec)}
        \operation{solve\_nonlinear(pvec, uvec, fvec)}
        \operation{apply\_linear(pvec, uvec, fvec, dpvec, duvec, dfvec)}
        \operation{solve\_linear(pvec, uvec, fvec, dpvec, duvec, dfvec)}
    \end{interface}

    \begin{class}{Component}{-7,-7}
        \implement{System}
        \attribute{pvec - input}
        \attribute{uvec - state vars \& outputs}
        \attribute{fvec - residual vars}
        \operation{add\_input(var\_name)}
        \operation{add\_output(var\_name)}
        \operation{add\_state(var\_name)}

    \end{class}

    \begin{class}{Group}{-1,-4.5}
        \implement{System}
        \attribute{non-linear solver}
        \attribute{linear solver}
        \attribute{workflow}
        \attribute{sequence}
        \attribute{exec_type}
        \operation{setup_sequence}
    \end{class}

    \begin{class}{Namespace}{6,-4.5}
        \inherit{Group}
        \attribute{name}
        \attribute{non-linear solver}
        \attribute{linear solver}
        \operation{connect(source, sink)}
    \end{class}

    \composition{Group}{}{1..*}{Component}
    \composition{Namespace}{}{1..*}{Component}
    \composition{Namespace}{}{1..*}{Group}

\end{tikzpicture}
\pagebreak


\section{Component API}

The basic unit of computational work in OpenMDAO is the \classname{Component}.
Componnets can contain implicit and explicit functions. Implicit functions are created
by adding state variables. Explicit functions are defined by adding outputs.


\subsection{Component Partial Derivatives}

Users building a component can choose to return a dictionary from linearize.
This is the simplest, and most compact way to define derivatives.

\begin{pyglist}[language=python]
    class SomeOtherComp(Component):
        def __init__(self):
            super(SomeComp,self).__init__()

            self.add_input('x1', size=(2,3))
            self.add_input('x2', val=3.4)

            self.add_state('y1', size=(2,3))
            self.add_output('z1', size=(2,3))

        def apply_nonlinear(self, pvec, uvec, fvec):

            # note that the fvec object needs to do a subtraction during this set
            fvec['z1'] = 3*pvec['x1'] + pvec['x2']
            fvec['y1'] = uvec['y1'] - pvec['x1']

        def linearize(self):
            self.J = {}
            self.J['z1','x1'] = 3
            self.J['z1','x2'] = 1

            self.J['y1'] = 1 # derivative of resid w.r.t state
            self.J['y1','x1'] = -1

            return J
\end{pyglist}

If the user wishes to implenent the linear operator directly, they can overload
the \method{apply\_linear} from the \classname{Component} base class.

\begin{pyglist}[language=python]
    class SomeComp(Component):
        def __init__(self):
            super(SomeComp,self).__init__()

            self.add_input('x1', size=(2,3))
            self.add_input('x2', val=3.4)

            self.add_state('y1', size=(2,3))
            self.add_output('z1', size=(2,3))

        def apply_nonlinear(self, pvec, uvec, rvec):

            rvec['z1'] = 3*pvec['x1'] + pvec['x2']
            rvec['y1'] = uvec['y1'] - pvec['x1']

        def linearize(self):
            pass # nothing meaningful to do here

        def apply_linear(self, pvec, uvec, dpvec, duvec, drvec, mode='fwd'):

            if mode="fwd":
                drvec['z1'] += 3*duvec['x1']
                drvec['z1'] += duvec['x2']
                drvec['y1'] += duvec['y1'] - duvec['x1']

            elif mode='adj':
                duvec['x1'] += 3*drvec['z1']
                duvec['x2'] += drvec['z2']
                duvec['y1'] += drvec['y1'] - drvec['x1']
\end{pyglist}

An alternate proposal is to do away with apply apply_linear as a separate method and
rely instead only on the returned value for J, where each entry in J can optional
allow either a dense matrix, a sparse matrix, or a linear operator. One very nice feature
of this is that it would allow users to mix and match a bit, if they wanted to.

\begin{pyglist}[language=python]
    class SomeComp(Component):
        def __init__(self):
            super(SomeComp,self).__init__()

            self.add_input('x1', size=(2,3))
            self.add_input('x2', val=3.4)

            self.add_state('y1', size=(2,3))
            self.add_output('z1', size=(2,3))

        def apply_nonlinear(self, pvec, uvec, rvec):

            rvec['z1'] = 3*pvec['x1']**2 + pvec['x2']
            rvec['y1'] = np.exp(uvec['y1']) - pvec['x1']

        def linearize(self):
            self.J = {}


            def fwd(self, pvec, uvec, dpmat, dumat, drmat, mode='fwd'):
                drvec['z1'] += 6*uvec['x']*duvec['x1']
            def rev(self, pvec, uvec, dpmat, dumat, drmat, mode='fwd'):
                duvec['x1'] += 6*uvec['x']*drvec['z1']
            self.J['z1','x1'] = {'fwd': fwd, 'rev': rev}

            self.J['z1','x2'] = 1

            def fwd(self, pvec, uvec, dpmat, dumat, drmat, mode='fwd'):
                drvec['y1'] += np.exp(uvec['y1'])*duvec['y1']
            def rev(self, pvec, uvec, dpmat, dumat, drmat, mode='fwd'):
                duvec['y1'] += np.exp(uvec['y1'])*drvec['y1']
            self.J['y1'] = {'fwd': fwd, 'rev': rev} # derivative of resid w.r.t state

            self.J['y1','x1'] = -1

            return J

\end{pyglist}

The system interface requires the \method{multi\_apply\_linear}  method which provides matrix
vector products on multiple vectors at once. \classname{Component} base implementation
for \method{multi\_apply\_linear} is just a for-loop over \method{apply\_linear}. But like
the \method{apply\_linear} method, users can overload it if they choose.

\begin{pyglist}[language=python]
    class SomeComp(Component):
        def __init__(self):
            super(SomeComp,self).__init__()

            self.add_input('x1', size=(2,3))
            self.add_input('x2', val=3.4)

            self.add_state('y1', size=(2,3))
            self.add_output('z1', size=(2,3))

        def apply_nonlinear(self, pvec, uvec, rvec):

            fvec['z1'] = 3*pvec['x1']**2 + pvec['x2']
            fvec['y1'] = uvec['y1'] - pvec['x1']

        def linearize(self):
            pass # nothing meaningful to do here

        def multi_apply_linear(self, pvec, uvec, dpmat, dumat, drmat, mode='fwd'):

            if mode="fwd":
                dumat['z1'] += 3*2*pvec['x1'].dot(dpmat['x1'])
                dumat['z1'] += dpmat['x2']
                dfmat['y1'] += dumat['y1'] - dpmat['x1']

            elif mode='adj':
                dpmat['x1'] += 3*2*pvec['x1'].dot(dumat['z1'])
                dpmat['x2'] += dumat['z1']

                dumat['y1'] += dfmat['y1']
                dpmat['x1'] -= dfmat['y1']
\end{pyglist}

\subsection{Grouping related variables}
It is often desireable to group sets of related variables together in a hierarchical
tree structure. However, OpenMDAO requires a totally flat variable space foreach component.
The solution to this issue is to establish a hierarchical naming convention of your choosing, and
utilize it in the creation of variables.

\begin{pyglist}[language=python]
    class SomeOtherComp(Component):
        def __init__(self):
            super(SomeComp,self).__init__()

            self.add_input('x.x1', size=(2,3))
            self.add_input('x.x2', val=3.4)

            self.add_state('y1', size=(2,3))
            self.add_output('z1', size=(2,3))

        def apply_nonlinear(self, pvec, uvec, fvec):

            uvec['z1'] = 3*pvec['x.x1'] + pvec['x.x2']
            fvec['y1'] = uvec['y1'] - pvec['x1']

        def linearize(self):
            self.J = {}
            self.J['z1','x.x1'] = 3
            self.J['z1','x.x2'] = 1

            self.J['y1'] = 1 # derivative of resid w.r.t state
            self.J['y1','x.x1'] = -1

            return J
\end{pyglist}

Conceptually, it is clear that there is now a group of variables collectively
identified by \texttt{x}, but the variable space in actuality remains completely
flat.

\section{Group: Execution Order}
All groups contain a \texttt{workflow} attribute which specifies the order its children will run in.
By default, the order children are added determines the \texttt{workflow}.
However, the user is free to redefine the \texttt{workflow} themselves if they choose.

Although the \texttt{workflow} guarantees the order the user defined children run in, it
does not define the full set of children a group will execute. The full set of
child systems that will actually be executed is defined by the \texttt{sequence}.

There are two ways in which the \texttt{sequence} could deviate from the \texttt{workflow}.
First, if a group's \textttt{exec\_type} attribute is set to \texttt{'auto'} (the default),
then that group will automatically break its children up into smaller parallel and serial sub-groups
based on the data-dependence-graph for its children. Second, any data transformations required by the
data-dependence-graph (e.g. unit conversions or connection expressions) will introduce additional
\classname{Component} instances to the sequence.

The \method{setup\_sequence} method is responsible for constructing the sequence for group with a
given workflow. \method{setup\_sequence} should be called recursively on all child groups inside a
containing group.

\section{Namespace: Variable Naming and Connections}
All variables have a fully qualified name, consisting of a complete path from the top most
\classname{Namespace} instance recursively down through the containing systems and including
the variable name itself.

\subsection{Justin's Idea}

\begin{pyglist}[language=python]

    class Compressor(NameSpace):
        def __init__(self):
            self.add('pressure_rise', PressureRise()) # has a variable PR
            self.add('ideal_flow', SetTotalSP())
            self.add('enthalpy_rise', EnthalpyRise())
            self.add('real_flow', SetTotalhP())
            self.add('power', Power())

            self.promote('real_flow.Tt', 'Fl_O.Tt')
            self.promote('real_flow.Pt', 'Fl_O.Pt')

    class EngineCycle(NameSpace):

        def __init__(self):
            self.add('hpc', Compressor())

            self.promote('hpc.pressure_rise.PR', 'PR')

\end{pyglist}

The fully qualified name for the \var{PR} variable in the \classname{PressureRise}
component , from within the \className{EngineCycle} namespace, is \var{hpc.pressure\_rise.PR}.
However \var{hpc.pressure\_rise.PR} was also promoted, and so the location in the data vector
that holds it can also be addressed by \var{PR} from within the \className{EngineCycle} namespace.

Similarly, \var{'real\_flow.Tt'} and \var{'real\_flow.Pt'} have been promoted into the
\classname{Compressor} namespace. So they they can be addressed within the \classname{EngineCycle}
namespace by \var{hpc.Fl\_O.Tt} and \var{hpc.Fl\_O.Pt}

All variable connection statements happen within an instance of \classname{Namespace}.

\subsection{John's Idea}

\begin{pyglist}[language=python]

    class Compressor(NameSpace):
        def __init__(self):
            self.add('pressure_rise', PressureRise(),
                global_inputs=('PR')) # has a variable PR
            self.add('ideal_flow', SetTotalSP())
            self.add('enthalpy_rise', EnthalpyRise())
            self.add('real_flow', SetTotalhP(),
                global_outputs=(('Tt','Fl_O.Tt'),('Pt', 'Fl_O.Pt'))
            self.add('power', Power())

    class EngineCycle(NameSpace):

        def __init__(self):
            self.add('hpc', Compressor())

            self.promote('hpc.pressure_rise.PR', 'PR')

\end{pyglist}


\section{Drivers (Optimizers and DOE)}
Note: Linear and non-linear solvers do not count as drivers

A \classname{Driver} is an iterative process that operates on a single \classname{System} contained inside it.

The \classname{Driver} base class run the \classname{System} it contains for a single iteration.
Other types of \classname{Driver} include Optimizers, CaseIteratorDriver, DesignOfExperiments, IterateUntil
SensitivityDriver, or Montecarlo.

Although \classname{Driver} does implement the \classname{System} interface it is not a regular system.
Within a hierarchy composed of  \classname{Component}, \classname{Group}, and \classname{NameSpace}
instances all variables are available to any system at any hierarchical level.However, an instance
of \classname{Driver} represents an opaque boundry across which only certain defined variables are allowed
to cross.

An instance of \classname{Driver} may be used as a \classname{Component} in a higher level analysis,
but that top level analysis will not know anything about the internal composition of the driver. For example,
if derivatives were requested from the driver, by the top level analysis, then the driver would need to
implement its own \method{linearize} (and possibly \method{apply\_linear} or \method{mult\_apply\_linear} methods).
The \classname{Driver} base class could provide a finite-difference based implementation.

Note: Driver might be a sub-class of Component, but its not totally clear yet.
\begin{itemize}
    \item How do you define variables? Probably with the same \method{add\_input}, \method{add\_state}, \method{add\_output} api.
    \item How do you get data from the system to the boundary variables? Maybe just have the user issue manaul assignments in
    a \method{post\_execute} or some other kind of hook.
\end{itemize}


\end{document}

\classname{Driver} instances
